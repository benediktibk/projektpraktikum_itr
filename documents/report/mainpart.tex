
%_________Mainpart__________________________________

\chapter{Theoretical Background}

\section{Bla1}
\label{sec:bla1}
\lipsum[1] 

\subsection{System Identification}
\label{subsec:system_identification}

\subsubsection{One Grasping Point}
Inertia of the system in the grasping Point P
\begin{equation}
	I_0 = 
	\begin{bmatrix}	
		I_{xx}	& I_{xy}	& I_{xz} \\
		I_{xy}	& I_{yy}	& I_{yz} \\
		I_{xz}	& I_{yz}	& I_{zz}
	\end{bmatrix}
\end{equation}
Vector from grasping point P to the center of gravity Q (or the other way round?)
\begin{equation}
	c = [c_x, c_y, c_z]^T
\end{equation}
Position of center of gravity
\begin{equation}
	p = [p_x, p_y, p_z]^T
\end{equation}
Angular velocity of the object
\begin{equation}
	\omega = [\omega_x, \omega_y, \omega_z]^T
\end{equation}
Input vector filled with forces and torques at the grasping point
\begin{equation}
	h = [F_x, F_y, F_z, M_x, M_y, M_z]^T
\end{equation}
System parameter vector
\begin{equation}
	\Theta = [m, m c_x, m c_y, m c_z, I_{xx}, I_{xy}, I_{xz}, I_{yy}, I_{yz}, I_{zz}]^T
\end{equation}
System parameter estimation vector $\hat \Theta$

Relation between $\Theta$ and $h$
\begin{equation}
	h(t) = \Phi(t) \Theta(t)
\end{equation}
with
\begin{equation}
	\Phi(t) = 
	\begin{bmatrix}
		\ddot p - g	& [\dot \omega \times] + [\omega \times] [\omega \times]	& 0 \\
		0			& [(g - \ddot p) \times]									& [. \dot \omega] + [\omega \times] [. \omega]
	\end{bmatrix}
\end{equation}

Initialization of Recursive Least Squares
\begin{equation}
	P(t_0) = \left( \Phi^T(t_0) \Phi(t_0) \right)^{-1}
\end{equation}
\begin{equation}
	\hat \Theta (t_0) = P(t_0) \Phi^T(t_0) h(t_0)
\end{equation}

Recursive Least Squares
\begin{equation}
	K(t) = P(t-1) \Phi^T(t) (I + \Phi(t) P(t - 1) \Phi^T(t))^{-1}
\end{equation}
\begin{equation}
	P(t) = (I - K(t) \Phi(t)) P(t - 1)
\end{equation}

\chapter{Modelling and Verification of Embedded Systems}

\section{Bla3}
\label{sec:bla3}
\lipsum[1] 
