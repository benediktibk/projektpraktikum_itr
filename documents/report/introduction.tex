
%_________Introduction__________________________________

\chapter{Introduction}
Robots conquer more and more our daily lifes. Despite that fact, they still have only very limited abilities. They are usually developed for one certain purpose and anything beyond that is impossible. To circumevent this drawback, as a first step the robots have to be able to explore their surroundings on their own. They must be able to identify objects and to measure them. This second task is the so-called load estimation, in which the robot measures the mass, the center of gravity and the intertia tensor of an unknown object. Due to limited mechanic abilities a robot often can not lift heavy objects and excited them well enough to identify these parameters.

During this internship we explored the possibility to incoroporate a human into the identification process. First of all, this enables the load estimation of heavier objects. Second, the human can provide additional excitation, which improves the estimation results. To coordinate the multiple agents we used vibrotactile wristbands, through which the robot can request additonal excitation by the human.

The result of this work is that the incorporation of a human agent improves load estimation results, although it might be difficult for an inexperienced user to provide the necessary exciation. In comparison to a load estimation carried out only by the robot the human-robot-team is able to identify heavier objects more accuracte. The second point holds only if the human excites the load in the necessary way, otherwise this has at least not a negative effect on the results. In the end the accuracy of the load estimation depends heavily on the exciation pattern.
