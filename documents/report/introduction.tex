\chapter{Introduction}
Robots conquer more and more our daily lifes. Despite that fact, they still have only very limited abilities. They are usually developed for one certain purpose, and anything beyond that is impossible. To circumvent this drawback, as a first step the robots have to be able to explore their surroundings on their own. They must be able to identify objects and to measure them. This second task is the so-called load estimation, in which the robot measures the mass, the center of gravity and the inertia tensor of an unknown object. A least-squares based algorithm that allows for the identification of load parameters using a robotic actuator is discussed in \cite{literaturstelle1} and \cite{literaturstelle2}. These algorithms recently have been extended towards a cooperative multi-robot estimation scheme as presented in \cite{literaturstelle3}. However, due to limited mechanic abilities one or multiple robots often are unable to lift and manipulate heavy objects. Especially robots used in academic research cannot track trajectories required to identify all unknown load parameters.

Within this project-laboratory, we explore the possibility to incorporate a human agent into the identification process. First of all, this enables the load estimation of heavier objects. Second, the human can provide additional excitation, which improves the estimation results. In order to coordinate the multiple agents we used vibrotactile wristbands, through which the robot can request additional excitation by the human.

The result of this work is that the incorporation of a human agent improves load estimation results, although it might be difficult for an inexperienced user to provide the necessary excitation. In comparison to a load estimation carried out only by the robot, the human-robot-team is able to identify heavier objects more accurate. The second point holds only if the human excites the load in the necessary way, otherwise this at least does not have a negative effect on the results. In the end the accuracy of the load estimation depends heavily on the excitation pattern.