\documentclass[student,noshadow]{ITRslides}
\usepackage{multimedia}

\usepackage[absolute,overlay]{textpos}
\renewcommand{\vec}[1]{\boldsymbol{#1}}
\addbibresource{ref.bib}
\graphicspath{{pics/}{logos/}}
\usepackage{psfrag}
\usepackage[percent]{overpic}
\usepackage{subcaption}
\usepackage{units}
\usepackage{tikz}
\usetikzlibrary{positioning,shapes}
\title{Object exploration using visual/haptic information by a human-robot team}
\presenter{Florian Wirnshofer, Benedikt Schmidt}

\supervisor{Denis Cehajic}
\typeofpres{Project Laboratory Cognitive Robotics and Control}


\newcommand*\MyBlue{%
  \item[\color{blue}\scalebox{0.9}{\textbullet}]}
\newcommand*\MyRed{%
  \item[\color{red}\scalebox{0.9}{\textbullet}]}
%\nocite{*}

\renewcommand{\vec}[1]{\boldsymbol{#1}} 
% Alle indizes in Normalschrift ausser Läuferindizeszs
\newcommand{\scr}[1]{\mathrm{#1}} 


%%%%%%%%%%%%%%%%%%%%%%%%%%%%%%%%%%%%%%%%%%%%%%%%%%%%%%%%%%%%%%%%%%%%%%%%%%%%%%%%

\begin{document}


\begin{frame}
    \titlepage
\end{frame}

%%%%%%%%%%%%%%%%%%%%%%%%%%%%%%%%%%%%%%%%%%%%%%%%%%%%%%%%%%%%%%%%%%%%%%%%%%%%%%%%
%INTRODUCTION , MOTIVATION
%%%%%%%%%%%%%%%%%%%%%%%%%%%%%%%%%%%%%%%%%%%%%%%%%%%%%%%%%%%%%%%%%%%%%%%%%%%%%%%%
\begin{frame}
	\frametitle{Content}
	\tableofcontents
\end{frame}

\section{Online Load Estimation}
\begin{frame}
	%Benedikt
	\frametitle{Task}
	\begin{block}{Goal}
			Human-Robot cooperative estimation of load uncertainties.
	\end{block}
	\vspace{2mm}
	\begin{block}{Key-Questions}
			\begin{itemize}
				\item Which sensor feedback is required from agents?
				\item How to exchange information between agents?
				\item How to fuse and process sensor feedback, resulting in a reliable load-identification?
			\end{itemize}	   
	\end{block}	
\end{frame}

\begin{frame}
	%Florian
	\frametitle{Online Load Estimation}
\end{frame}

\begin{frame}
	%Florian
	\frametitle{Cooperative Online Load Estimation}
\end{frame}

\begin{frame}
	%Florian
	\frametitle{Persistent Excitation}
	\simpleblock{
	\begin{small}
		\begin{center}
			RLS convergence prerequisites
		\end{center}
	\end{small}
	}
	\vspace{1cm}	
	\begin{itemize}
		\item Reference trajectory must be persistently exciting(PE)
		\item Non-zero acceleration of EEF in 6-DoF \cite{literaturstelle3}
	\end{itemize}
	\vspace{1cm}
	\textsc{Challenge}: Satisfaction of actuator limits, especially when trying to identify big objects.
\end{frame}

\section{Signal Processing}
\begin{frame}
	%Florian
	\frametitle{Acquisition}
\end{frame}

\begin{frame}
	%Florian
	\frametitle{Processing}
\end{frame}

\section{Simulation Results}
\begin{frame}
	%Benedikt
	\frametitle{Estimation Results without Noise}
\end{frame}

\begin{frame}
	%Benedikt
	\frametitle{Estimation Results with Noise}
\end{frame}

\section{Outlook}
\begin{frame}
	%Benedikt
	\frametitle{Remaining Issues}
\end{frame}
\begin{frame}
	%Benedikt
	\frametitle{Time Schedule}
	\begin{tikzpicture}[node distance=0.2cm]
		\node[shape=circle,fill=tum_blue] (A) at (0, 0) {};
		\node[right=of A.east,anchor=west] {Get familiar with the topic and the hardware};
		\node[anchor=west] (DateA) at (0.2, -0.4) {19.11.2014};
		\draw[tum_blue,thick] (DateA) -- (A);
		\node[shape=circle,fill=tum_blue] (B) at (0, -1) {};
		\node[right=of B.east,anchor=west] {Implement load identification with one grasping point};
		\node[anchor=west] (DateB) at (0.2, -1.4) {5.12.2014};
		\draw[tum_blue,thick] (DateB) -- (B);
		\node[shape=circle,fill=tum_blue] (C) at (0, -2) {};
		\node[right=of C.east,anchor=west] {Implement load identification with more than one grasping point};
		\node[anchor=west] (DateC) at (0.2, -2.4) {19.12.2014};
		\draw[tum_blue,thick] (DateC) -- (C);
		\node[shape=circle,fill=tum_blue] (D) at (0, -3) {};
		\node[right=of D.east,anchor=west] {Trigger additional excitation by the human through the wrist band};
		\node[anchor=west] (DateD) at (0.2, -3.4) {16.01.2015};
		\draw[tum_blue,thick] (DateD) -- (D);
		\draw[tum_blue,very thick] (0, 0) -- (0, -3);
	\end{tikzpicture}
\end{frame}

\appendix
\begin{frame}
	\frametitle{References}
	\printbibliography
\end{frame}

\end{document}