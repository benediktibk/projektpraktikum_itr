\documentclass[student,noshadow]{ITRslides}
\usepackage{multimedia}

\usepackage[absolute,overlay]{textpos}
\renewcommand{\vec}[1]{\boldsymbol{#1}}
\addbibresource{ref.bib}
\graphicspath{{pics/}{logos/}}
\usepackage{psfrag}
\usepackage[percent]{overpic}
\usepackage{subcaption}
\usepackage{units}
\usepackage{tikz}
\usetikzlibrary{positioning,shapes}
\title{Object exploration using visual/haptic information by a human-robot team}
\presenter{Florian Wirnshofer, Benedikt Schmidt}

\supervisor{Denis Cehajic}
\typeofpres{Project Laboratory Cognitive Robotics and Control}


\newcommand*\MyBlue{%
  \item[\color{blue}\scalebox{0.9}{\textbullet}]}
\newcommand*\MyRed{%
  \item[\color{red}\scalebox{0.9}{\textbullet}]}
%\nocite{*}

\renewcommand{\vec}[1]{\boldsymbol{#1}} 
% Alle indizes in Normalschrift ausser Läuferindizeszs
\newcommand{\scr}[1]{\mathrm{#1}} 


%%%%%%%%%%%%%%%%%%%%%%%%%%%%%%%%%%%%%%%%%%%%%%%%%%%%%%%%%%%%%%%%%%%%%%%%%%%%%%%%

\begin{document}


\begin{frame}
    \titlepage
\end{frame}

%%%%%%%%%%%%%%%%%%%%%%%%%%%%%%%%%%%%%%%%%%%%%%%%%%%%%%%%%%%%%%%%%%%%%%%%%%%%%%%%
%INTRODUCTION , MOTIVATION
%%%%%%%%%%%%%%%%%%%%%%%%%%%%%%%%%%%%%%%%%%%%%%%%%%%%%%%%%%%%%%%%%%%%%%%%%%%%%%%%
\begin{frame}
	\frametitle{Content}
	\tableofcontents
\end{frame}

\section{Task}
\begin{frame}
	%Florian
	\frametitle{Task}
	\begin{block}{Goal}
			Human-Robot cooperative estimation of load uncertainties.
	\end{block}
	\vspace{2mm}
	\begin{block}{Key-Questions}
			\begin{itemize}
				\item Which sensor feedback is required from agents?
				\item How to exchange information between agents?
				\item How to fuse and process sensor feedback, resulting in a reliable load-identification?
			\end{itemize}	   
	\end{block}	
\end{frame}

\section{Load Identification}
\begin{frame}
	\frametitle{Load Identification - State of the Art}

	\begin{columns}
		\centering
		 	\begin{column}{0.5\textwidth}
				\begin{overpic}[width=0.8\textwidth]{stateoftheart.eps}
								\put(60,40){\color{red}{\small $\vec{c}$}}
								\put(70,20){\small $\vec{p}$}
								\put(25,-3){\small $\scr{W}$}
								\put(10,45){\small $\scr{CoM}$, $\vec{q}$}
			    \end{overpic}
		 	\end{column}
		 		
		 	\begin{column}{0.5\textwidth}
			Model:\\ \cite{literaturstelle2}\\
			\vspace{0.1cm}
			$\vec{F} = f\left(\vec{\ddot{p}},\vec{\omega},\vec{\dot{\omega}},\vec{c},m\right)$\\
			$\vec{\tau} = f\left(\vec{\ddot{p}},\vec{\omega},\vec{\dot{\omega}},\vec{c},\vec{I},m\right)$\\
			\vspace{0.4cm}
			$m$ Object mass\\
			$\vec{I}$  Object inertia tensor
		 	\end{column}
	\end{columns}
	 		\vspace{0.5cm}
			Estimation-Parameters: \\
			$\vec{\Theta} = [m, m c_\scr{x}, m c_\scr{y}, m c_\scr{z}, I_\scr{xx}, I_\scr{xy}, I_\scr{xz}, I_\scr{yy},I_\scr{yz}, I_\scr{zz}]^\scr{T}$ \\
		\vspace{0.3cm}
			RLS-Regressor: \\
	$\vec{\Phi}(k) = 
	\begin{bmatrix}
		\vec{\ddot p} - \vec{g}	& [\vec{\dot \omega} \times] + [\vec{\omega}\times] [\vec{\omega} \times]	& 0 \\
		0			& [(\vec{g} - \vec{\ddot p}) \times]									& [.  \vec{\dot \omega}] + [\vec{\omega} \times] [. \vec{\omega}]
	\end{bmatrix}$
	
\end{frame}

\begin{frame}
	\frametitle{Load Identification with Human}
		\begin{columns}
		\centering
		 	\begin{column}{0.5\textwidth}
	\begin{overpic}[width=0.8\textwidth]{humanident.eps}
		\put(60,40){\color{red}{\small $\vec{c}$}}
		\put(70,20){\small $\vec{p}$}
		\put(25,-3){\small $\scr{W}$}
		\put(20,35){\small $\scr{CoM}$, $\vec{q}$}
		\put(-6,55){\color{green}{\small H}}
		\put(-6,48){\small $\vec{h}$}
    \end{overpic}
		 	\end{column}	 		
		 	\begin{column}{0.5\textwidth}
			Model:\\
			\vspace{0.1cm}
			$\sum\vec{F} = f\left(\vec{\ddot{p}},\vec{\ddot{h}},\vec{\omega},\vec{\dot{\omega}},\vec{c},m\right)$\\
			$\sum\vec{\tau} = f\left(\vec{\ddot{p}},\vec{\ddot{h}},\vec{\omega},\vec{\dot{\omega}},\vec{c},\vec{I},m\right)$\\
			\vspace{0.4cm}
			$m$ Object mass\\
			$\vec{I}$  Object inertia tensor
		 	\end{column}
	\end{columns}
	\vspace{1cm}
	\begin{itemize}
		\item Redefine regressor $\vec{\Phi}(k)$.
		\item Replace recursive least-squares (RLS) with recursive total least-squares (RTLS) to handle errors
		noise and disturbances affecting the acceleration and angular velocity signals.\\ \cite{literaturstelle1}
	\end{itemize}
\end{frame}

\section{Excitation}
\begin{frame}
	\frametitle{Persistent Excitation}
	\simpleblock{
	\begin{small}
		\begin{center}
			RLS / TRLS convergence prerequisites
		\end{center}
	\end{small}
	}
	\vspace{1cm}	
	\begin{itemize}
		\item Reference trajectory must be persistently exciting(PE)
		\item Non-zero acceleration of EEF in 6-DoF \cite{literaturstelle3}
	\end{itemize}
	\vspace{1cm}
	\textsc{Challenge}: Satisfaction of actuator limits, especially when trying to identify big objects.
\end{frame}
\begin{frame}
	%Benedikt
	\frametitle{Cooperative Excitation}	
	\begin{itemize}
		\item Problem: Excitation only by the robot may not be optimal to receive a good and fast estimation of the load kinematics.
		\item Solution: Robot can give orders to the human through a wrist band
		\item How will a human react to a certain stimulus?		
	\end{itemize}
\end{frame}

\section{Tools}
\begin{frame}
	\frametitle{Tools}
	Hardware:
	\begin{itemize}
		\item Cobot: Two arms with seven joints each
		\item Force and torque sensors at the endeffector and the grapsing points of the human
		\item Motiontracking system for positions
		\item Video cameras at the robot for visual input
	\end{itemize}
	\vspace{0.5cm}
	Implementation:
	\begin{itemize}
		\item Matlab Simulink
		\item Sensor input through SFBReceive-Blocks
		\item Actuation through SFBSend-Blocks
	\end{itemize}
\end{frame}

\begin{frame}
	\frametitle{Noise on Sensors}
	Noisy sensors:
	\begin{itemize}
		\item Force and torque sensors
		\item Pose acquired by motion tracking system 
	\end{itemize}
	\vspace{1cm}
	Solution:
	\begin{itemize}
		\item RLS to circumvent noise in forces and torques
		\item RTLS to consider noise in poses and velocities
	\end{itemize}
\end{frame}

\section{Time Schedule}
\begin{frame}
	\frametitle{Milestones}
	\begin{tikzpicture}[node distance=0.2cm]
		\node [shape=circle,fill=tum_blue] (A) at (0, 0) {};
		\node [right=of A.east,anchor=west] {Get familiar with the topic and the hardware};
		\node [shape=circle,fill=tum_blue] (B) at (0, -1) {};
		\node [right=of B.east,anchor=west] {Implement load identification with one grasping point};
		\node [shape=circle,fill=tum_blue] (C) at (0, -2) {};
		\node [right=of C.east,anchor=west] {Implement load identification with more than one grasping point};
		\node [shape=circle,fill=tum_blue] (D) at (0, -3) {};
		\node [right=of D.east,anchor=west] {Trigger additional excitation by the human through the wrist band};
		\draw[tum_blue,very thick] (0, 0) -- (0, -3);
	\end{tikzpicture}
\end{frame}

\begin{frame}
	\frametitle{Time Schedule}
	\begin{tikzpicture}[node distance=0.2cm]
		\node[shape=circle,fill=tum_blue] (A) at (0, 0) {};
		\node[right=of A.east,anchor=west] {Get familiar with the topic and the hardware};
		\node[anchor=west] (DateA) at (0.2, -0.4) {19.11.2014};
		\draw[tum_blue,thick] (DateA) -- (A);
		\node[shape=circle,fill=tum_blue] (B) at (0, -1) {};
		\node[right=of B.east,anchor=west] {Implement load identification with one grasping point};
		\node[anchor=west] (DateB) at (0.2, -1.4) {5.12.2014};
		\draw[tum_blue,thick] (DateB) -- (B);
		\node[shape=circle,fill=tum_blue] (C) at (0, -2) {};
		\node[right=of C.east,anchor=west] {Implement load identification with more than one grasping point};
		\node[anchor=west] (DateC) at (0.2, -2.4) {19.12.2014};
		\draw[tum_blue,thick] (DateC) -- (C);
		\node[shape=circle,fill=tum_blue] (D) at (0, -3) {};
		\node[right=of D.east,anchor=west] {Trigger additional excitation by the human through the wrist band};
		\node[anchor=west] (DateD) at (0.2, -3.4) {16.01.2015};
		\draw[tum_blue,thick] (DateD) -- (D);
		\draw[tum_blue,very thick] (0, 0) -- (0, -3);
	\end{tikzpicture}
\end{frame}

\appendix
\begin{frame}
	\frametitle{References}
	\printbibliography
\end{frame}

\end{document}