\documentclass[student,noshadow]{ITRslides}
\usepackage{multimedia}

\usepackage[absolute,overlay]{textpos}
\renewcommand{\vec}[1]{\boldsymbol{#1}} 
\addbibresource{ref.bib}
\graphicspath{{pics/}{logos/}}
\usepackage{psfrag}
\usepackage[percent]{overpic}
\usepackage{subcaption}
\usepackage{units}
\title{Object exploration using visual/haptic information by a human-robot team}
\presenter{Florian Wirnshofer, Benedikt Schmidt}

\supervisor{Denis Cehajic}
\typeofpres{Project Laboratory Cognitive Robotics and Control}


\newcommand*\MyBlue{%
  \item[\color{blue}\scalebox{0.9}{\textbullet}]}
\newcommand*\MyRed{%
  \item[\color{red}\scalebox{0.9}{\textbullet}]}
%\nocite{*}

\renewcommand{\vec}[1]{\boldsymbol{#1}} 
% Alle indizes in Normalschrift ausser Läuferindizeszs
\newcommand{\scr}[1]{\mathrm{#1}} 


%%%%%%%%%%%%%%%%%%%%%%%%%%%%%%%%%%%%%%%%%%%%%%%%%%%%%%%%%%%%%%%%%%%%%%%%%%%%%%%%

\begin{document}


\begin{frame}
    \titlepage
\end{frame}

%%%%%%%%%%%%%%%%%%%%%%%%%%%%%%%%%%%%%%%%%%%%%%%%%%%%%%%%%%%%%%%%%%%%%%%%%%%%%%%%
%INTRODUCTION , MOTIVATION
%%%%%%%%%%%%%%%%%%%%%%%%%%%%%%%%%%%%%%%%%%%%%%%%%%%%%%%%%%%%%%%%%%%%%%%%%%%%%%%%
\begin{frame}
	\frametitle{Content}
	\tableofcontents
\end{frame}

\section{Task}
\begin{frame}
	%Florian
	\frametitle{Task}
	\begin{flushleft}
	\begin{overpic}[width=0.6\textwidth, grid, tics=10]{stateoftheart.eps}
					\put(60,40){\color{red}{\small $\vec{c}$}}
					\put(70,20){\small $\vec{p}$}
					\put(25,-2){\small $\scr{W}$}
    \end{overpic}
	\end{flushleft}	
\end{frame}

\section{Load Identification}
\begin{frame}
	\frametitle{Load Identification - State of the Art}
\end{frame}

\begin{frame}
	\frametitle{Load Identification with Human}
\end{frame}

\section{Excitation}
\begin{frame}
	\frametitle{Persistent Excitation}
\end{frame}

\begin{frame}
	%Benedikt
	\frametitle{Cooperative Excitation}
	\begin{itemize}
		\setlength\itemsep{0.5em}
		\item Robot can give orders to the human through a wrist band
		\item Target: Get more information about the load
		\item Problem: How will a human react to a certain stimulus?
	\end{itemize}
\end{frame}

\section{Tools}
\begin{frame}
	\frametitle{Tools}
	\begin{itemize}
		\setlength\itemsep{0.5em}
		\item Cobot: Two arms with seven joints each
		\item Force and torque sensors at the endeffector and the grapsing points of the human
		\item Motiontracking system for positions
		\item Video cameras at the robot for visual input
		\item Implementation in Matlab Simulink
	\end{itemize}
\end{frame}

\section{Time Schedule}
\begin{frame}
	\frametitle{Milestones}
	\begin{enumerate}
		\item Get familiar with the system
		\item Implement load identification with one grasping point
		\item Implement load identification with more than one grasping point
		\item Trigger additional exciation by the human through the wrist band
	\end{enumerate}
\end{frame}

\begin{frame}
	\frametitle{Time Schedule}
	\begin{enumerate}
		\item Get familiar with the system: 19.11.2014
		\item Implement load identification with one grasping point: 5.12.2014
		\item Implement load identification with more than one grasping point:  19.12.2014
		\item Trigger additional exciation by the human through the wrist band: 16.01.2015
	\end{enumerate}
\end{frame}

\appendix
\begin{frame}
	\frametitle{References}
	\printbibliography
\end{frame}





\end{document}